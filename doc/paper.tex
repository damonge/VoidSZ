\documentclass[twocolumn,amsfont,amssymb,amsmath, showpacs,balancelastpage, nofootinbib]{revtex4-1}
\pdfoutput=1

\usepackage{graphicx}
\usepackage{dcolumn}
\usepackage{bm}
\usepackage{amssymb,amsmath,bm}  
\usepackage{color}
\usepackage[colorlinks,linkcolor=red,citecolor=blue,urlcolor=blue ]{hyperref}
\usepackage{multirow}
\usepackage[utf8]{inputenc}
\usepackage{balance}
\usepackage{enumitem}
\usepackage{lipsum}
\newcommand{\nv}{\hat{\bf n}}
\newcommand{\kalo}{Karhunen-Lo\`{e}ve\,}
\newcommand{\jcap}{JCAP}
\newcommand{\mnras}{MNRAS}
\newcommand{\aap}{A\&A}
\newcommand{\aaps}{A\&AS}
\newcommand{\apjs}{ApJS}
\newcommand{\apjl}{ApJL}
\newcommand{\aj}{Astron. Journal}
\newcommand{\pasp}{Publications of the ASP}
\newcommand{\nar}{New Astronomy Review}
\newcommand{\procspie}{Proceedings of the SPIE}
\newcommand{\physrep}{Physics Reports}

\begin{document}
  \title{The Sunyaev-Zel'dovich effect around voids}
  \author{All of us$^1$}
  \affiliation{$^{1}$Miskatonic University}

  \begin{abstract}
    \lipsum[0]
  \end{abstract}

  \maketitle

  \section{Introduction}\label{sec:intro}
    \lipsum[1]
    
  \section{Theoretical model}\label{sec:theory}
    The tSZ signal caused by a particular structure at redshift $z$ is given by
    \begin{equation}
      y(\theta)=\frac{\sigma_T}{m_e\,c^2}\int\frac{dr_\parallel}{1+z}P_e\left(\sqrt{r_\parallel^2+r_\perp^2}\right),
    \end{equation}
    where $\sigma_T$ and $m_e$ are the Thomson scattering cross section and the electron mass, $P_e(r)$ is
    the electron pressure profile of the structure, $r_\parallel$ and $r_\perp\equiv \chi(z)\theta$ are
    the longitudinal (parallel to the line of sight) and transverse comoving distances from the structure,
    $\chi(z)$ is the comoving distance to redshift $z$ and $\theta$ is the angular separation from the
    center of the projected structure.

    The tSZ signal around voids can therefore be predicted by estimating their expected excess electron pressure
    profile. We do so here by connecting the void density profile, which can be estimated directly from the data,
    with $P_e(r)$ using the so-called ``effective universe'' approach. This method is spelled out in Appendix
    \ref{app:effu}, and has been previously used in analyses of environmental effects on halo abundances \cite{2003MNRAS.344..715G,2004ApJ...605....1G,2009MNRAS.394.2109M,2015MNRAS.447.2683A}). In short, one can
    associate the void underdensity $\delta(r)$ with a set of effective cosmological parameters $\Omega_X(r)$,
    which can then be used to estimate any quantity in the void as its background value in that effective
    cosmology.
    
    The problem therefore reduces to estimating the background free electron pressure for a given set of
    cosmological parameters. Assuming the main contribution to the total tSZ signal comes from the hot gas in
    dark matter haloes, the total electron pressure at a point ${\bf r}$ is given by the sum of the contributions
    from all haloes:
    \begin{align}
      P_e({\bf r})=\int d{\bf x}^3\,dM\,n(M,{\bf x})P_e(|{\bf x}-{\bf r}|,M),
    \end{align}
    where $n(M,{\bf x})$ is the number density of haloes of mass $M$ (i.e. the position-dependent halo mass
    function), with pressure profile $P_e(r,M)$. The background contribution to the electron pressure is
    therefore found by taking the ensemble average of the equation above:
    \begin{equation}\label{eq:pe_bg}
      \langle P_e \rangle=\int dM\,n(M)\frac{4\pi}{3}\int dr\,r^2\,P_e(r,M).
    \end{equation}
    
    To summarize, the process to estimate the void's electron pressure profile is therefore as follows:
    \begin{enumerate}
      \item Estimate the void's over-density profile $\delta(r)$.
      \item At a given $r$, relate $\delta(r)$ to a set of effective cosmological parameters $\Omega_X(r)$.
      \item The void's electron pressure at that $r$ is then computed using Eq. \ref{eq:pe_bg} as the
            background electron pressure for the corresponding effective cosmology parameters. Note that,
            in this equation, both the mass function and the halo pressure profile depend on $\Omega_X$.
    \end{enumerate}




  \section{Data}\label{sec:data}
    \lipsum[3]

  \section{Results}\label{sec:results}
    \lipsum[4]

  \section{Discussion}\label{sec:discussion}
    \lipsum[4]
  
    \cite{2017JCAP...07..014H}
    \cite{2017MNRAS.469..787P}
    \cite{2014PhRvL.112y1302H}
    \cite{2014MNRAS.442..462S}
    \cite{2016IAUS..308..542N}

\section*{Acknowledgements}
  \lipsum[5]
  
\bibliography{paper}

\appendix
\onecolumngrid
\section{The effective-universe approach to void-related quantities}\label{app:effu}
  \lipsum[6]

\end{document}
